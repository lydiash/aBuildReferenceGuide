
\documentclass{article}
\usepackage[utf8]{inputenc}
\usepackage{relsize}

\title{So You Want to do Materials Research:\\[0.02em]\smaller{}a
guide to aBuild and the skills you need to use it}
\author{Lydia Harris and Eli Harris}

\begin{document}

\maketitle

\section{Bash Commands}

First things first, you need to learn to navigate your command
line. Macs have a built in command line (terminal), but on Windows you will need
to download one first. The Ubuntu app works great.

\begin{table}
        \begin{center}
                \caption{Bash commands and what they mean}
                \label{bashcommands}
                \begin{tabular}{l|r}
                        \textbf{Command} & \textbf{What it does}\\
                        \hline
                        ls & list contents of current directory \\
                        ls -a & show hidden files too \\
                        mkdir directory & make a new directory \\
                        cd directory & change directory \\
                        cd .. & go back a directory \\
                        cd ../.. & go back two directories \\
                        cd ~ & go to your root \\
                        pwd & print working directory \\
                        ~/ & means your root \\
                        . & means the current directory \\
                        cp file/to/copy where/it/goes/newName & copy
                        a file \\
                        cp file/to/copy . & copy a file to current
                        directory without \\
                         & changing the name \\
                        cp files/* . & copy all the files in a
                        directory \\
                         & to the current directory \\
                        cp -r directory new/directory & copy a
                        directory recursively \\
                        rm file/to/remove & remove a file \\
                        rmdir directory & remove a directory \\
                        rm -rf directory & blow away a directory
                        permanently \\
                        mv file/to/move where/it/goes/newName & moves
                        or renames a file \\
                        man command & show the manual for a command \\
                        cat file/one file/two \textgreater new/file & concatonate
                        two or more files into a new file \\
                        history & shows a history of your commands \\
                        less file/to/see & shows one page of a file \\
                         & space turns the page q quits \\
                        head file/to/see & see the first page of a file \\
                        head -n 8 file/to/see & see the first 8 lines
                        of a file \\
                        tail file/to/see & see the last page of a file
                        \\
                        tail -n 10 file/to/see & see the last 10 lines
                        of a file \\
                        grep keyword file/to/search & search a file
                        for a keyword and \\
                         & print all the lines with that keyword to the screen \\
                        history | grep keyword & search your history
                        for a keyword \\
                        grep keyword file/to/search \textbar wc -l & count
                        the occurences of lines with a keyword \\
                        command \textbar less & pipe the output of a command
                        to less \\
                        command \textgreater \textgreater file & append the output of a
                        command to a file \\
                        command \textgreater file & writes the output of the
                        command to a file \\
                        !command & executes the most recent command
                        that \\
                         & starts with the letters you typed \\
                        echo & print something to the screen \\
                 \end{tabular}
        \end{center}
\end{table}
                        
\section{Initialization Process}
To start working there are a series of steps and proceedure that must
be followed. The first step is to obtain an account to the supercomputer--likely
marylou. After setting up and account and logging on, you will need to
make several directories and download aBuild.  
\subsection{aBuild Download}
1. Become familiar with git --see git tutorial
\subsection{Make directories}
You will need to build the following directories:\\
\indent 1. .\verb|\|codes\verb|\|aBuild\\
\indent 2. \verb|\|bin\\
\indent 3. \verb|\|environments\\
\indent 2. \verb|\|Species-system (i.e. AuCu)

\subsubsection{bin directory}
\section{Training Process}                      
Once you have set up your system you are ready to start training the
model to your specific system. The python module aBuild is designed to
automate the train of the model process. This section will guild you
through the different commands needed to train the model.\\

The commands start with python ``builder.py **YMAL**" and then the
various tags are used to step through the training process. The python
command is only used for the tags denoted by the `-'.
**YMAL** is the yml file without the .yml extension.

\begin{table}
        \begin{center}
                \caption{aBuild steps}
                \label{bashcommands}
                \begin{tabular}{l|r}
                  \textbf{Step} & \textbf{Description}\\
                  \hline
                  -enum & Looks for possible crystal structures to
                          enumerate through\\
                  -write -rgk & Build the training set folder and run
                                the getKPoints script\\
                  -setup\_train & Pull data from VASP folders, Build
                                 train.cfg and pot.mtp\\
                  qsub* jobscript\_setup.sh & mlp train: needs
                                             train.cfg, pot.mtp;
                                             generates: Potential.mtp
                  -set & 
                                                      
                 \end{tabular}
        \end{center}
\end{table}

*For Marylou use sbatch instead of qsub

\end{document}