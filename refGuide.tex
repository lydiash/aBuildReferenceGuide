
\documentclass{article}
\usepackage[utf8]{inputenc}
\usepackage{relsize}

\title{So You Want to do Materials Research:\\[0.02em]\smaller{}a
guide to aBuild and the skills you need to use it}
\author{Lydia Harris and Eli Harris}

\begin{document}

\maketitle

\section{Bash Commands}

First things first, you need to learn to navigate your command
line. Macs have a built in command line (terminal), but on Windows you will need
to download one first. The Ubuntu app works great. See
\ref{bashcommands} for a list of helpful commands. 

\begin{table}
        \begin{center}
                \caption{Bash commands and what they mean}
                \label{bashcommands}
                \begin{tabular}{l|r}
                        \textbf{Command} & \textbf{What it does}\\
                        \hline
                        ls & list contents of current directory \\
                        ls -a & show hidden files too \\
                        ls -altr & see the last changes made to the \\
                         & files in a directory \\
                        mkdir directory & make a new directory \\
                        cd directory & change directory \\
                        cd .. & go back a directory \\
                        cd ../.. & go back two directories \\
                        cd ~ & go to your root \\
                        pwd & print working directory \\
                        ~/ & means your root \\
                        . & means the current directory \\
                        cp file/to/copy where/it/goes/newName & copy
                        a file \\
                        cp file/to/copy . & copy a file to current
                        directory without \\
                         & changing the name \\
                        cp files/* . & copy all the files in a
                        directory \\
                         & to the current directory \\
                        cp -r directory new/directory & copy a
                        directory recursively \\
                        rm file/to/remove & remove a file \\
                        rmdir directory & remove a directory \\
                        rm -rf directory & blow away a directory
                        permanently \\
                        mv file/to/move where/it/goes/newName & moves
                        or renames a file \\
                        man command & show the manual for a command \\
                        cat file/one file/two \textgreater new/file & concatonate
                        two or more files into a new file \\
                        history & shows a history of your commands \\
                        less file/to/see & shows one page of a file \\
                         & space turns the page q quits \\
                        head file/to/see & see the first page of a file \\
                        head -n 8 file/to/see & see the first 8 lines
                        of a file \\
                        tail file/to/see & see the last page of a file
                        \\
                        tail -n 10 file/to/see & see the last 10 lines
                        of a file \\
                        grep keyword file/to/search & search a file
                        for a keyword and \\
                         & print all the lines with that keyword to the screen \\
                        history | grep keyword & search your history
                        for a keyword \\
                        grep keyword file/to/search \textbar wc -l & count
                        the occurences of lines with a keyword \\
                        command \textbar less & pipe the output of a command
                        to less \\
                        command \textgreater \textgreater file & append the output of a
                        command to a file \\
                        command \textgreater file & writes the output of the
                        command to a file \\
                        !command & executes the most recent command
                        that \\
                         & starts with the letters you typed \\
                        echo & print something to the screen \\
                  
                 \end{tabular}
        \end{center}
\end{table}

\section{Emacs}

Emacs is a text editor that has a bunch of cool shortcuts you can
learn to make editing documents super easy. On a Mac you can download
Aquamacs, which uses the same commands as Emacs but you can click with
your mouse. \ref{emacsChart} Has a chart of basic Emacs
commands. \ref{AquamacsChart} has a list of Aquamacs specific things. 

\begin{table}
        \begin{center}
                \caption{Emacs commands and what they mean}
                \label{emacsChart}
                \begin{tabular}{l|r}
                  \textbf{Command} & \textbf{What it does}\\
                  \hline
                  emacs path/to/file & enter emacs editor for
                                             existing file \\
                   & or creates new file with that name \\
                  ctrl+x ctrl+c y & save and quit a file \\
                  ctrl+x ctrl+c n & quit without saving \\
                  ctrl+w & cut a line \\
                  ctrl+y & paste a line \\
                  ctrl+k & kills the contents of a line \\
                  ctrl+k ctrl+k & kills a whole line \\
                 \end{tabular}
        \end{center}
\end{table}

\begin{table}
        \begin{center}
                \caption{Extra things in Aquamacs}
                \label{AquamacsChart}
                \begin{tabular}{l|r}
                  \textbf{Command} & \textbf{What it does}\\
                  \hline
                  ctrl+x ctrl+f & find and open a file (at the \\
                   & bottom of the screen) \\
                 \end{tabular}
        \end{center}
\end{table}

\section{Bash Loops}

Here's a table \ref{bashloops} of basic bash loops and logic, and a
basic example \ref{loopexample} that relates to what we do.

\begin{table}
        \begin{center}
                \caption{Loops in Bash}
                \label{bashloops}
                \begin{tabular}{l|r}
                  \textbf{Command} & \textbf{What it does}\\
                  \hline
                  \verb|for i in {1..100}| & for 100 iterations\\
                  \verb|do|
                  \verb|command $i| & do this thing \verb|$i| \\
                  \verb|done| & references the index \\
                  \hline
                  \verb|if [ condition ]| & check the condition \\
                  \verb|then| & if it's true \\
                  \verb|command| & do this \\
                  \verb|else| & if it's not \\
                  \verb|command| & do this \\
                  \verb|fi| & \\
                  \hline
                  \verb|if [ -e file ] |& check if a file exists \\
                 \end{tabular}
        \end{center}
      \end{table}
      
\begin{table}
        \begin{center}
                \caption{Example}
                \label{loopexample}
                \begin{tabular}{l|r}
                  \textbf{Command} & \textbf{What it does}\\
                  \hline
                  \verb|for i in {1..100}| & for 100 times \\
                  \verb|do|
                  \verb|cd E.$i| & enter the directory named
                                   \verb|E.#|\\
                  \verb|if [ -e KPOINTS ]| & if KPOINTS doesn't exist
                  \\
                  \verb|echo $i| & print the directory number \\
                  \verb|getKPoints| & run the getKPoints script \\
                  \verb|fi|
                  \verb|cd ..| & go back one directory \\
                  \verb|done|
                 \end{tabular}
        \end{center}
\end{table}


%%%% DON'T KNOW WHERE THIS BELONGS IN THIS DOCUMENT BUT I'M GONNA TYPE
%%%% IT UP SO THAT WE HAVE IT AND CAN PLACE IT LATER

How to make and use a virtual environment to run python interactively, table \ref{venv}.

\begin{table}
        \begin{center}
                \caption{Example}
                \label{venv}
                \begin{tabular}{l|r}
                  \textbf{Command} & \textbf{What it does}\\
                  \hline
                  \verb|mkdir environments| & make a directory to \\
                                   & hold your environments \\
                  \verb|cd environments| & go to that directory \\
                  \verb|python -m venv name_of_env| & make the
                                                      environment \\
                  \verb|emacs .bash_profile| & edit your
                                               \verb|.bash_profile|\\
                  \verb|function work on {| & add these lines to it \\
                  \verb| source ~/environments/$1/bin/activate| & \\
                  \verb|}|
                  ctrl+x ctrl+c y & save and quit the editor \\
                  source \verb|.bash_profile| & update your
                                                \verb|.bash_profile|\\
                  \verb|workon name_of_env| & enters your environment
                  \\
                  ctrl+d & exits your environment \\
                \end{tabular}
        \end{center}
\end{table}

\section{Basics of Github}

This section is by no means comprehensive. In fact, there's probably a
lot of things missing. If you'd like to actually get good at Github,
you have a lot more work to do. But here's some basics.

The basic idea of Github is that several people can work on the same
code at the same time. You can have a copy of the code on your machine
to work on, and you can push your changes to the main copy of the
code. Most people, however, don't have their repositories open to the
public to edit. They will have to clear any suggested changes that you
push. But as they make edits, you can pull their copy down to your
machine. Github will warn you if there are any changes you have made
to the code that will be overwritten by copying their changes. It's a
powerful code sharing tool, but takes a little getting used to. 

First you need to go make a Github account. Then you can fork (make a
copy of) whatever repository of code that you want to copy and
edit. If you don't want to make edits to the code yourself, you don't
have to make a fork, you can just copy directory from the original
repository each time.
To give your computer (or a remote computer) access to your repositories
on Github, add your computer's public SSH key to Github. How to do
this:
On your command line execute the following command to copy
your public ssh key:
\begin{verbatim}
pbcopy < ~/.ssh/id_rsa.pub
\end{verbatim}
If this didn't work, it means you don't have a public key and will
need to generate one.
%%%%%WE NEED TO FIGURE OUT HOW TO DO THIS AGAIN. CAN'T FIND THE
%%%%%WEBPAGE WE RECEIVED OUR INSTRUCTIONS FROM
Now you will be able to paste this key into your account on Github. To
do this, go to Settings>SSH and GPG keys>New SSH key. Make sure you
name it something intuitive (e.g. My Mac, Marylou, etc.).

Now you can make a copy of this code on your machine. It's probably a
good idea to make a directory to hold all the stuff you're going to
copy. Usually this is in \verb|~/codes/|. If you just want to make a
copy of the code and don't expect any changes to be made to it, use
the following command:
\begin{verbatim}
git clone git@github.com:user/repository.git
\end{verbatim}
If you made your own fork and  expect to make changes to the code, add
a master copy of your fork to your machine using the following command:
\begin{verbatim}
git remote add master git@github.com:user/repository.git
\end{verbatim}
If you want to be able to add someone else's version of the code to
your machine, the convention is to call their repository
\verb|upstream|. You can add access to their repository to your
machine with the following command:
\begin{verbatim}
git remote add upstream git@github.com:user/repository.git
\end{verbatim}

                        
\section{Initialization Process}
\subsection{git download}
\subsection{Make directories}
\subsubsection{bin directory}
\section{Training Process}                      



\end{document}