
\documentclass{article}
\usepackage[utf8]{inputenc}
\usepackage{relsize}
\usepackage{float}
\usepackage{longtable}
\usepackage[dvipsnames]{xcolor}
\usepackage{graphicx} %for pictures
\usepackage{eufrak} %for fancy neighborhood n
\usepackage{indentfirst} %to indent the first paragraph after a title
\usepackage{enumitem} %itemize spacing options
\usepackage{textcomp} %for tildesssssss?????!!!
\usepackage{nameref} %for referencing unnumbered sections by name
\usepackage[section]{placeins} %float barriers
\setlist[itemize]{nosep, leftmargin=.5cm, topsep=0pt, partopsep=0pt}
\setlist[enumerate]{leftmargin=.5cm}

\title{So You Want to do Materials Research:\\[0.02em]\smaller{}a
guide to aBuild and the skills you need to use it}
\author{Lydia Harris and Eli Harris}

\begin{document}
\maketitle
\begin{abstract}
 The python module aBuild is meant to automate the process of
 building an MTP model for a paticular system. The documents presented
 in this folder/book are meant to be a reference guide to bash, git,
 aBuild, etc. Disclaimer: we wrote this to help ourselves remember
 these things, and as a favor to Brother Nelson. We reserve the right
 for some of it to be incomplete or confusing at times. If you have
 questions, don't be afraid to ask us or Brother Nelson. 
\end{abstract}

\section{Getting Started}
To do computational research you will need to learn what the command
line is and some bash commands--see \nameref{sec:bash}. Go
learn your commands it will make your life easier!! 

Now that you know the basics of the command line environment it is
time to utilize those skills to log into the supercomuter. If you have
not yet made an account, go to marylou.byu.edu click on request an
account. Your faculty mentor (most likely Brother Nelson) will need to
approve your account. The whole process may take a couple of
days. Once you have an account, if you are using Mac or Linux simply
open the terminal and use the ssh command (see
\nameref{sec:bash}.). If you have a Windows machine you have several
options: you can use the Ubuntu app, Windows power shell, Windows
terminal, etc.   

\subsection{Make directories}
Once logged into the supercomputer you will need to build the
following directories (see \nameref{sec:bash}):
\begin{itemize}
  \item{\verb|\home\codes\aBuild|}
  \item{\verb|\home\bin|}
  \item{\verb|\home\environments|}
  \item{\verb|\home\system-species| (i.e. AgPt)}
\end{itemize}

If you don't know what system you're gonna study, talk with Brother
Nelson before you make your system-species folder. He'll help you find
one to study. 

\subsection{Download aBuild}
Now it is time to download aBuild. The python module aBuild is meant
to automate the process of building a MTP model for a paticular
system. The module is continuously undergoing improvements. To have
version control we use git. If you are unfamiliar with git go read
\nameref{sec:github}. Git is an extremely helpful tool and used across
many different organizations. It's a skill you should add to your
resume once you are familiar with it. Once you've learned how to,
fork/download aBuild from https://github.com/lancejnelson/aBuild (see
table \ref{git} for a quick refresher). 

\subsection{Virtual Environment/bash\_profile}
A convenient way to install and run software is in a virtual
environment. Because aBuild is a software package, it needs to be
installed for you to be able to run it. We will install it in a python
virtual environment. This is how you will create a virtual environment
and install aBuild:

\begin{center}
  \begin{longtable}{||p{7cm}|p{4cm}||} %longtable so it wraps %
    % pages   
    \caption{Making a virtual environment and installing aBuild}
    \label{bashcommands}
    \\ \hline
    \textbf{Command} & \textbf{Description}\\ \hline \hline
    \endhead
    % make the previous page footer
    \hline
    \multicolumn{2}{||c||}
    {\tablename\ \thetable\ -- \textit{Continued on next
        page}} \\ \hline
    \endfoot
    %make the last footer
    \hline
    \endlastfoot
    %contents of the table
    cd environments & go into that directory \\
    python $-$m venv name\_of\_env	& make the environment\\
    emacs .bash\_profile & edit your .bash\_profile \\
    \verb|function workon {| & add these lines to it: \\
    \verb|source ~/environments/$1/bin/activate| &	\$1 refers to
                                                   the name of \\
    \verb|}| & your virtual environment\\	
    ctrl$+$x ctrl$+$c y & quit the editor\\
    source .bash\_profile & update your .bash\_profile \\
    workon name\_of\_env & enters your environment \\
    python install -e \texttildelow\slash codes/aBuild & install
      aBuild in your virtual environment \\
    ctrl$+$d & exits your environment \\
  \end{longtable}
\end{center}

Since we used the .bash\_profile to make the virtual environment work,
I'll talk a little more about that now. Your .bash\_profile holds the
settings that are applied each time you log in to the
supercomputer. You can load software modules, change settings, and a
lot of other things. Some helpful additions to your .bash\_profile are
given here (actually I basically just copied and pasted my entire
.bash\_profile here), although these were the current software modules
on Marylou in Dec 2018, and I reserve the right to have old versions
of the modules loaded, since it is probably not Dec 2018 anymore.

\vspace{5mm}
%\noindent
\begin{itemize}
  \item{export PATH$=$ \$ PATH:\texttildelow \slash bin}  
  \item{PS1$=$``\textbackslash u:\textbackslash w\textbackslash \$ "}
  \item{module load libfabric}
  \item{module purge}
  \item{module load compiler\_intel\slash 13.0.1}
  \item{module load gdb\slash 7.9.1}
  \item{module load gcc\slash 6.4}
  \item{module load python\slash 3.6}
  \item{export MAKESTRX$=$\texttildelow\slash bin\slash makestr.x}
  \item{export GETKPTS$=$\texttildelow\slash bin\slash getKPoints}
  \item{export ENUMX$=$\texttildelow\slash bin\slash enum.x}
  \item{export F90$=$ifort}
  \item{export HISTSIZE$=$100000}
  \item{set completion-ignore-case on}
  \item{function workon \{\\
      source\texttildelow /environments/\$1/bin/activate\\
      \} }
  \item{alias hh$=$'history $|$ grep '}
\end{itemize}

\subsection{Copy other software packages}
You'll need enum.x (.x means it's an executable file), makestr.x, and
getKPoints. There may be copies of these in the group folder that you
can copy to your bin. Ask Brother Nelson for help with this if there's
not. 

\section{Run aBuild: Training Process}                      
Now that you have your account all set up, you are ready to start
training the model to your specific system. This section will guide
you through the different commands needed to train the model. 

Your folder will need builder.py and a yaml file in it to run any of
these commands. Copy these from 
\verb|~/codes/aBuild/aBuild/templates/master.yml| and
\verb|~/codes/aBuild/aBuild/scripts/builder.py|. Name your yaml
something intuitive (e.g. if you're studying the silver gold system,
maybe name it ``AgAu,'' etc..). Now you'll need to edit the yaml to
make sure everything in there matches your system (such as the title,
species, root, potcar directory, potcar versions, potcar setups,
mindistance, concs, nconfigs, sizes, etc..). If you need help
deciphering the yaml, go see the example.yml file in the
aBuildReferenceGuide repository on git. It has a bunch of comments to
help you understand what's going on, although you don't really need to
understand what's going on to get started. 

Now you can start building your model. You can see the steps in table
\ref{algorithm} below. aBuild commands start with
``\verb|python builder.py **YML**|" and then some tag(s). This prefix
is only used for the tags denoted in this table by the `-'.  

\begin{center}
  \begin{longtable}{||p{4cm}|p{7cm}||} %longtable so it wraps %
    % pages   
    \caption{Algorithm steps and their descriptions.}
    \label{algorithm}
    \\ \hline
    \textbf{Step} & \textbf{Description}\\ \hline \hline
    \endhead
    % make the previous page footer
    \hline
    \multicolumn{2}{||c||}
    {\tablename\ \thetable\ -- \textit{Continued on next
        page}} \\ \hline
    \endfoot
    %make the last footer
    \hline
    \endlastfoot
    %contents of the table
    -enum & enumerates the crystalline structures up to sizes specified in your
            yaml \textcolor{red}{--run interactively}\\ 
    -setup\_relax & Builds to-relax.cfg and relax.ini; runs calc-grade
                    \textcolor{red}{--run interactively}\\
    qsub* jobscript\_relax.sh & mlp relax: needs to-relax.cfg, pot.mtp, and
                                relax.ini; generates: relaxed.cfg,
                                unrelaxed.cfg, and candidates.cfg \textcolor{red}{--job
                    submission, parallel, 10-30 cores, 6-30 hrs}\\
    -setup\_select\_add & Concatenates all of the candidate.cfg\_\#,
                          selection.log\_\#, relaxed.cfg\_\# and
                          unrelaxed.cfg\_\# into one file
                          each. relaxed.cfg file should get bigger and
                          bigger with each iteration. Also builds a
                          submission script. \textcolor{red}{--run interactively}\\
    qsub* jobscript\_select.sh & mlp select-add: generates:
                              new\_training.cfg; needs: train.cfg,
                              candidate.cfg \textcolor{red}{--job
                              submission, single core, 1-4 hrs}\\
    -add & builds A folders in training set and creates jobscript \textcolor{red}{--run interactively}\\
    qsub* jobscript\_vasp.sh & runs vasp calculations for
                                           the selected configurations
                                            \textcolor{red}{--array
                                            job, 6-30 hrs}\\
    -setup\_train & Pulls data from VASP folders, builds
                    train.cfg and pot.mtp \textcolor{red}{--run interactively} \\
    qsub* jobscript\_train.sh & mlp train: needs train.cfg, pot.mtp;
                                generates: Potential.mtp \textcolor{red}{--job
                    submission, parallel, 10-20 cores, 6-12 hrs}\\
    %go back to step -setup\_train & Repeat until model is fully
    %                                trained, i.e. all structures relax.\\
    go back to step -setup\_relax & Repeat until model is fully
                                    trained, i.e. all structures relax.\\
  \end{longtable}
\end{center}
*For Marylou use sbatch instead of qsub
**YML** is the yaml file without the .yml extension.

The training process will begin with an empty training set. This will
cause the relaxation to terminate for each structure on the first
iteration. It will take several iterations before the model is able to
relax all the configurations.

The mlp relax step tries to relax the structures in the to relax
set. If it extrapolates too much, it stops relaxing it, and adds the
structure to a preselected set.

The mlp select add step chooses from structures in candidate.cfg the
structures that best fill the ``missing'' configuration space.

The mlp train step tries fits a ``line'' to the training data it is
given. I say ``line'' and not line because it is a non linear problem,
but if it helps you to think of it like a line, do that. 

\subsection{Other helpful MTP/aBuild commands}
These might come in handy at some point
\begin{center}
  \begin{longtable}{||p{4cm}|p{7cm}||} %longtable so it wraps %
    % pages   
    \caption{aBuild commands and description}
    \label{bashcommands}
    \\ \hline
    \textbf{Command} & \textbf{Description}\\ \hline \hline
    \endhead
    % make the previous page footer
    \hline
    \multicolumn{2}{||c||}
    {\tablename\ \thetable\ -- \textit{Continued on next
        page}} \\ \hline
    \endfoot
    %make the last footer
    \hline
    \endlastfoot
    % contents of the table
    -status & prints a status report for Vasp calculations \\
    -report & creates data report file from completed Vasp
                           calculations \\
    -report -file path/to/file & creates data report file from the
                           configurations contained in the file specified \\
    -chull -file path/to/file & creates a convex hull from the data
                           report file specified \\
    mlp mindist file.cfg & prints the global minimum distance of the
                           atoms to eachother in a .cfg file. Also
                           adds the mindist attribute to structures in
                           the .cfg file\\
    mlp calc-grade pot.mtp train.cfg train.cfg temp.cfg & creates
                           state.mvs, a file needed for relaxation \\
    
  \end{longtable}
\end{center}

\section{Theory}
In this section we discuss what the algorithms are doing or the
theory behind the computations. You don't really need to know any of
this to get started working on this project, but if you're curious,
(as you should be at some point, you're a scientist, after all) here's
some info for you.

The main idea is to create a MTP model from Vasp calculations. The
Vasp calculations are ab initio or first principle calculations that
use DFT and/or DFT+U depending on the specific system. The MTP then is
trained by optimization, fitting the coefficients of the basis
functions (see section \ref{sec:MTP}:MTP for more details). After the
model has been trained it attempts to relax the atoms to their happy
place. If it cannot relax the atoms and still accurately predict the
energy, forces, and stresses, the model then selects more
configurations to add to the training set. These configurations are
then evaluated using Vasp and the cycle continues until the model is
able to relax all of the configurations.

The motivation of training a MTP model comes from the bottleneck
caused by Vasp calculations. In searching for configurations to create the
convex hull, the configuration space for varying concentrations is so vast
that using DFT calculation becomes impractical. This is due to the amount of
time the calculations take. By using an MTP training model we are able
to explore much more of the configuration space in signicantly less
time. This allows us to potentially discover new configurations of
a paticular system.

DISCLAIMER: Some of this gets hard to understand. We hope that this
document will be a nice introduction without too much scary stuff, but
with that being said, don't go scaring yourself off by jumping into all
of this too early. Each section has a statement that says ``stop
reading here if you're not ready.'' Listen to these statements. 


\subsection{MTP} \label{sec:MTP}
MTP is an acronym for moment tensor potentials. It is a basis
expansion, and it involves a bunch of tensors. For the purpose of
doing this research, you don't really need to know a whole lot about
it, except that it's a way to represent crystalline structures that is
systematically improvable (meaning you can add more basis functions to
get a better and better representation of the crystal) unlike
classical potentials, and it is an off-lattice model (it isn't
confined to some parent structure, the atoms can be located anywhere
with respect to eachother), unlike cluster expansion. Basically it's a
brand spanking new (published just last year), way better basis for
crystal configurations than the previous methods (classical potentials
and cluster expansion, mentioned earlier). Figure \ref{fig:2d} is a
good cartoon visualization of what we're trying to accomplish with
this model. 

\begin{figure}[h]
  \centering
  \includegraphics[scale = .4]{configVsEnergy}
  \caption{ Simple 2-dimensional visualization of configuration
    space vs. energy with a best fit ``line''. Each configuration is on
    the x-axis with it's corresponding energy on the y-axis. In
    reality this graph would be N+1 dimensional, where N is the
    number of basis functions, $B_{\alpha} \left(\mathfrak{n} \right)
    $.}
  \label{fig:2d}
\end{figure}

This is your quitting point, but in case the previous paragraph isn't
enough for you, here's a quick summary of the MTP basis, with a little
math, but not very much explanation of the math. Sorry about that. In
this github repository we will also include Bro. Nelson's report that
has a really good explanation of the MTP basis, including a sample
problem that can help you understand it if you want to.

The moment tensor potential (MTP) basis is a set of orthogonal basis
functions given by: 
\begin{equation} \label{eq:basis}
  V \left(\mathfrak{n} \right) = \sum\limits_{\alpha}
  \xi_{\alpha}B_{\alpha} \left(\mathfrak{n} \right)
\end{equation}

The MTP basis can be used to represent crystalline configuration
space. A crystalline structure can be evaluated on this basis to an
arbitrary precision. A simple analogy to this kind of evaluation is
the Fourier transform, where higher and higher frequencies can be
added to make a better fit to any function. 

The basis functions $B_{\alpha} \left(\mathfrak{n} \right) $ of the
MTP basis depend on the set of moment tensor descriptors:
\begin{equation} \label{eq:descriptors}
  M_{\mu,\nu}\left(\mathfrak{n}_i\right) = \sum\limits_{j} f_{\mu}
  \left( \left| \mathbf{r}_{ij} \right| ,z_i,z_j \right)
  \underbrace{\mathbf{r}_{ij}\otimes...\otimes\mathbf{r}_{ij}}_{\nu
     times}
\end{equation}
These descriptors are dependent on the immediate neighborhood,
$\mathfrak{n}_i$, of the $i$th atom, within some $R_{cut}$, as shown
in Figure \ref{fig:neighborhood}.  Each atom in the neighborhood of
the $i$th atom introduces four degrees of freedom to the energy
contribution, $V_i$. The total energy, $E$, depends on each energy 
contribution, $V_i$. The four degrees of freedom are the three
coordinates in Euclidean space of the separation between atoms $i$ and
$j$, $r_{ij}$, and a discrete variable, $z_j$, that represents the
species of the neighboring atom. %These moment tensor descriptors are
%tensors of rank $J$, where $J$ is the number of atoms in the
%neighborhood. 

\begin{figure}[h]
  \centering
  \includegraphics[scale=1.2]{neighborhood.jpg}
  \caption{The $i$th atom's neighborhood is made up of each atom
    within some $R_{cut}$ of itself. The total energy E is made up of
    the contributions from individual neighborhoods. The energy
    contribution, $V_i$, of neighborhood $\mathfrak{n}_i$ depends on
    the separation between atoms $i$ and $j$, $r_{ij}$, and a discrete
    variable, $z_j$, that represents the species of the atom in the
    neighborhood (I or II in this illustration)
    \cite{gubaev2019accelerating}. }
  \label{fig:neighborhood}
\end{figure}

The $f_{\mu}\left( \left| \mathbf{r}_{ij} \right| ,z_i,z_j
\right)$ term in Equation \ref{eq:descriptors} is given by:

\begin{equation} \label{eq:someequation}
  f_{\mu} \left( \mathbf{\rho} ,z_i,z_j \right) =
  \sum\limits_{k}c_{\mu,z_i,z_j}^{\left( k \right)}Q^{\left( k \right)} \left(
    \mathbf{\rho}
  \right)
\end{equation}
where
\begin{equation}  \label{eq:cheby}
  Q^{\left(k\right)} \left( \mathbf{\rho} \right) =
  T_k\left( \mathbf{\rho} \right)\left( R_{cut} - \mathbf{\rho}
  \right)^2
\end{equation}
In Equation \ref{eq:cheby}, $T_k\left(  \mathbf{\rho} \right)$ are the
Chebyshev polynomials on the interval $\left[ R_{min},R_{cut}\right]$.

The $\mathbf{r}_{ij}\otimes...\otimes\mathbf{r}_{ij}$ terms in
Equation \ref{eq:descriptors} contain angular information about the
neighborhood $\mathfrak{n}_i$ and are tensors of rank $\mathbf{\nu}$.
The basis functions $B_{\alpha} \left(\mathfrak{n} \right) $ are made
up all of possible contractions of any number of
$M_{\mu,\nu}\left(\mathfrak{n}_i\right)$ that result in a scalar. The
maximum depth of these calculations is chosen, with more levels
providing more accuracy. This attribute makes the basis a
systematically improvable functional form, similar to including more
frequencies in a Fourier transformation.

The $\xi_{\alpha}$ and $c_{\mu,z_i,z_j}^{\left( k \right)}$ terms in
Equations \ref{eq:basis} and \ref{eq:someequation} are parameters that must
be optimized. A quasi-Newton optimization technique is used to fit
these to the data provided by the training set.  The configuration
space created by these basis functions is $N$ dimensional, where $N$
is the number of basis functions the crystal was evaluated on. The
optimization can be visualized in 2 dimensions, ``configuration'' vs
energy, shown in Figure \ref{fig:2d}, where the appropriate terms are
chosen to minimize the error of a best fit ``line'' to the training
data.


If you would like to know more, see ref
\cite{gubaev2019accelerating}. There are several other less
understandable (for undergraduates) papers that talk about the basis,
in refs \cite{podryabinkin2017active,gubaev2018machine,
  shapeev2016moment}. 


\subsection{DFT}
In quantum mechanics, the Schr\"{o}dinger equation must be solved to
find the energy of a system. Because of the size of many body
problems, it is impossible to solve the Schr\"{o}dinger equation
exactly for the system. This leads scientists to an approximation
called density functional theory (DFT). It is an ab-initio
calculation, or a first-principles calculation, if you've ever heard
those terms before. Density functional theory is based on the
assertion that the ground state energy of a system is a unique
functional (function of a function) of the electron density (which is
a function) \cite{PhysRev.136.B864}.

This is your quitting point. If you'd like to know more, go ahead and
keep on reading!

In this energy functional mentioned earlier, called the Kohn-Sham
equation \cite{kohn1965self}, every term can be known exactly except
the exchange-correlation (XC) functional.  

The XC functional can be approximated in many different ways,
including the Local Density Approximation (LDA), which assumes the
electronic density behaves locally like a homogenous electron
gas. Another approach is the Generalized Gradient Approximation (GGA),
which is similar to LDA but includes the local gradient, and other
derivative methods of these two basic methods.

Because the Kohn-Sham equation depends on the electronic density and
is also used to find the density, an iterative approach that
converges to be self consistent is used to solve for the energy of the
system.  DFT calculations are done in k-space, and sample points in
the first Brillouin zone called k-points are chosen and appropriately
weighted to replace the integrals in the Kohn-Sham equation. These are
the KPOINTS files in your Vasp folders.

\begin{figure}
  \includegraphics[scale = .5]{PAW}
  \caption{Comparison of atomic wave functions of Mn using the PAW
    method (solid line) with the exact result (bullets) for a given
    energy and angular momentum. Shown also are their differences
    magnified by a factor of 10 (dash-dotted line), and their pseudo
    wave functions (dashed line) \cite{PhysRevB.50.17953}. }
  \label{fig:MnPAW}
\end{figure} 

The method of pseudopotentials is often used to reduce the
computational load of DFT calculations. This method approximates the
inner electrons as a ``frozen core'' that place the electrons
surrounding them in an effective potential. The two common methods of
this approximation are the projector augmented-wave (PAW) method and
the UltraSoft PseudoPotential (USPP) method. The PAW method allows
all-electrons orbitals to be reconstructed from the pseudo-orbitals,
as shown in Figure \ref{fig:MnPAW}. 

Vasp stands for the Vienna Ab initio Simulation Package
\cite{kresse1996software}. This is the package we use to do DFT. Vasp
requires the user to choose a method for approximating the XC
functional. Vasp also prefers the use of the PAW method, but a
specific PAW potential must be chosen. These are the POTCARS you
specified in the yaml.

This is pretty much all I understand about DFT. If you'd like to know
more, there's a pretty good lecture series on Youtube:
https://youtu.be/vJkNv095Aj8, and a good book you could read
\cite{kitchin2008modeling}. 


\subsubsection{DFT+U}

With modeling large atoms (such as uranium), there can be some
difficulties with getting the DFT calculations to converge to a
``correct'' total energy. You probably won't use the method I'm about
to talk about, so you probably don't need to read this section unless
you've talked to Brother Nelson about modeling large atoms, or you're
morbidly curious. Read on, if you want.

This effect happens because the traditional treatment of electrons in
DFT calculations allows the Coulomb repulsion to scatter the
electrons, when in large atoms, the $d$ and $f$ electrons are strongly
correlated and localized. Because the energy of the system is
dependent upon the electron density, an incorrect density will often
predict an energy that is too high. 
 
To remedy the traditional treatment of electrons in large atoms, DFT+U
should be used. DFT+U is also known as LDAU. Without the addition of
the U parameter to DFT calculations, the calculation may converge,
but it will likely converge to a non-physical solution. 

The recommended method to ensure DFT+U converges to the
``correct'' total energy is to follow a ramping scheme discussed in ref
\cite{meredig2010method} that begins with a U parameter of 0 and ramps
up to 4.5 (in steps of 1, e.g. $0\Rightarrow 1\Rightarrow 2\Rightarrow
3\Rightarrow 4\Rightarrow 4.5$), using the charge density calculated in
the previous iteration. This helps ensure convergence to a ``true''
total energy. 

These are the settings in the INCAR that must be used to employ DFT+U:
\begin{itemize}
  \item{ENCUT = 550 }
  \item{LWAVE = False }
  \item{ LDAU = True }
  \item{ LDAUTYPE = 1 }
  \item{ LDAUL = 3 -1 -1 }
  \item{ LDAUU = \# 0 0 (\# changes with ramping scheme)}
  \item{ ICHARG = 1 (after the first iteration of ramping scheme)}
  \item{ LDAUJ = 0.51 0 0 }
  \item{ ISMEAR = -5 }
  \item{ LMAXMIX = 6 }
  \item{ ISIF = 2 }
  \item{ NSW = 0 }
\end{itemize}

\subsection{Optimization Routines}
Optimization has to do with finding the best choice of some possible
options. That's a really broad concept, but with relation to what we
are doing, we want to minimize the error of a best fit line. We create
a best fit ``line'' of energy vs. structure (remember figure
\ref{fig:2d}?) This is a really simplified explanation of what the
algorithm is actually doing, because a structure is decomposed into a
set of $N$ basis functions, and each basis function has a rank $n$ tensor
(where $n$ is the order of the system, e.g. binary, ternary) that also
has to be optimized with it, so it's not a linear system. 

The optimization scheme used to train the model is called BFGS. It
stands for Broyden-Fletcher-Goldfarb-Shanno. And now we're at your
quitting point. Feel free to keep reading if you're very curious (good
for you):

The BFGS algorithm is called a Quasi-Newton method, and is
based on the Newton method of optimization, which uses the Taylor
expansion of a function out to two derivatives, takes the partial with
respect to $ \Delta x $ and iteratively searches for the value of x that
makes the resulting function equal 0. This is the same as finding
where the derivative of a function equals 0. For the Newton method, we
need to find the second derivative to make this equation. For a
multivariable function, the second derivative is the Hessian
matrix. This matrix can be expensive to calculate, and thus
Quasi-Newton methods were developed, which approximate the Hessian in
one way or another. The BFGS algorithm requires an initial guess for 
the Hessian (usually an identity matrix of appropriate dimension) and
iteratively approximates a new one as it solves for the $x$ that
minimizes the function. The equation for the Hessian matrix that the
BFGS algorithm uses is as follows:

\begin{equation}
  B_{k+1}= B_k + \frac{y_k y_k^T}{y_k^T s_k} - \frac{B_k s_k s_k^T
    B_k^t}{s_k^T B_k s_k}
\end{equation}

\noindent
with $ y_k = \nabla f ( \, x_{k+1} ) \, - \nabla f ( \, x_k ) \, $ and $ s_k =
x_{k+1}-x_k $.

The algorithm essentially finds a step direction (direction of
steepest descent), finds an acceptable step size (backtracking line-search
algorithm), takes the step, and then solves for the approximate
Hessian matrix until the new $x$ value and the old one are close enough
to each other, according to some epsilon value.

The backtracking line-search algorithm start with a maximum step size,
and makes it smaller by some factor $ \tau \in ( \, 0,1 ) \, $ until it
satisfies what is called the Armijo-Goldstein condition, as follows:

\begin{equation}
  f ( \, x + \alpha p ) \, \leq \alpha c m
\end{equation}

\noindent
where $ m = p^T \nabla f ( \, x ) \, $ and $ c \in ( \, 0,1 ) \, $
is some control parameter.

Are you happy you read to the end of this section? 

\subsection{Convex Hull}
This is actually an important section. Go ahead and read the whole
thing if you get this far.

A convex hull is a cool mathematical tool, but it also has physical
importance. See figure \ref{fig:AgAuAFLOW} for a visualization. The
convex hull has some hints to what it is in it's name (for once a
useful name for something). It is constructed of the lowest energy
structures of a system that can be connected with a line that is
convex. It looks like the hull of a boat. The breaking points of the
convex hull represent the ground state structures of a system. (If you
didn't know, you can't actually make a crystalline structure with any
concentration of materials. For example, there is such thing as
$\mathrm{UO}_2$ and $\mathrm{U}_3\mathrm{O}_8$, but there is no such
thing as $\mathrm{U}_3\mathrm{O}_2$).

\begin{figure}[t!]
  \centering
  \includegraphics[scale=.3]{AgAuAFLOW.png}
  \caption{Convex hull of Ag-Au system as determined by 294
    high-throughput ab initio calculations
    \cite{curtarolo2012aflowlib}. }
  \label{fig:AgAuAFLOW}
\end{figure}

I lied to you. This is your quitting point, don't bother reading this unless you're doing a
ternary system (which you very well may be doing...). A three body system's convex hull
is the same thing mathematically, but is maybe a bit harder to
visualize because to see all of it you need 3 dimensions.  But you can
represent it in 2 dimensions, like in figure \ref{fig:3dchull}.

\begin{figure}[h]
  \centering
  \includegraphics[scale=1]{3dchull.jpg}
  \caption{Convex hull of Convex hull of the Co-Nb-V system
    constructed by MTP in the Co-rich region
    \cite{gubaev2019accelerating}.  }
  \label{fig:3dchull}
\end{figure}

% \subsection{}

\FloatBarrier

\section*{Appendix A: VASP files and what they do} \label{sec:vaspinput} %%%TODO
Pretty self explanatory name. If you want to learn a bit more about
VASP, this is a good section to read.

In order to run any Vasp calculation, you need a POSCAR, POTCAR,
INCAR, PRECALC, and KPOINTS. The rest of these files are output
files. There are a couple other output files I haven't mentioned
here. 
%I'll have to come back and add some stuff in here later.... This might
%be better if it wasn't an appendix and just a section in the regular
%ref guide. 

\subsection*{POSCAR}
This is the ``POSition'' car. Not really sure who came up with the
naming convetion, but whatever \verb|\_(-.-)_/|.
In this file there is title. There is also the lattice parameter,
which is what you multiply all the lattice vectors by to get the
cartesian coordinates for the lattice vectos, which are the three
lines that follow. Then there are the atom counts for each species (in
reverse alphabetical order), and then the coordinate system. This can
be ``D'' for direct, meaning that you multiply the first number of the
basis vector by the first lattice vector, the second by the second,
and the third by the third, then add up the x-coordinates, the
y-coordinates, and the z-coordinates to find the cartesian coordinate
of each atom within the unit cell. The ``C'' stands for cartesian
coordinates, so each number is just an x, y, or z component of the
basis atom's position. Next come the basis vectors, in either direct
or cartesian coordinates. 

If you didn't already know, the lattice vectors are the repeating unit
of a crystalline structure. If you slide the origin to the end of one
lattice vector, you will end up on the same position (as in, the same position
but in a neighboring unit cell). The basis vectors are the positions
of all the atoms in the unit cell. Usually there is one basis vector
with a value of (0,0,0), meaning that the lattice vectors lie in the
middle of one of the atoms. 

\subsection*{INCAR}
This is is ``INput'' car. This file has all the settings for the vasp calculations. In
\nameref{sec:VASPsettings} I talk about some of the settings we've
used in past works. They're probably safe to use, but you might want
to consult Brother Nelson about this. 

\subsection*{PRECALC}
This is the input file for k point generation. The only thing you need
to touch in here is mindistance, unless you find yourself in some very
dire circumstances. See \nameref{sec:VASPsettings} for an explanation
of such dire circumstances...

\subsection*{POTCAR}
This is the ``POTential'' car. It has the psuedopotentials in it. You
can \verb|grep TITEL POTCAR| to make sure that the correct atom types
are in here.   

\subsection*{KPOINTS}
This is generated by the getKPoints script. You probably don't ever
really have to worry about it unless you forget to generate it. 

\subsection*{OUTCAR} 
This is the ``OUTput'' car. It has some of the output in it. You can
\verb|grep TOTEN OUTCAR|  to see the total energy. If the energy has
converged, you will see ``\verb|free  energy|'' with two spaces.

\subsection*{CONTCAR}
This is an output file. If you let the atoms relax, this is the new POSCAR. Kinda. It just
specifies the new positions of the atoms. Also I'm not sure what the
``CONT'' is. 

\subsection*{CHGCAR}
Not sure, maybe the ``CHarGe'' car? Ask Brother Nelson

\subsection*{OSZICAR}
This has how many iterations Vasp has done in it. I'm not sure what
other purpose it serves, and I'm also not sure what the ``OSZI'' is. 


\section*{Appendix B: Bash Commands} \label{sec:bash}

To do any of this work, you need to learn to navigate your command
line. Macs and Linux have a built in command line (terminal). On
Windows, there are several options:
\begin{enumerate}
  \item{Windows Powershell. See:
      https://docs.microsoft.com/en-us/windows-
      server/administration/windows-commands/powershell
      for instructions on how to configure your pc to use it.}
  \item{The Ubuntu app--simply search for it in your windows store.}
  \item{Windows recently came out with Windows Terminal as a 
      central place for all the command line userfaces.}
  \item{If you know any other apps, feel free to use them}
\end{enumerate}
Your command line allows you to communicate with your machine by
typing. The language you use is called bash, and if you want to make a
script to execute bash commands, you call it a shell script. See Table
\ref{bashcommands} for a list of helpful commands.  

\begin{center}
  \begin{longtable}{||p{5.5cm}|p{5.5cm}||} %longtable so it wraps pages
    \caption{Bash commands and what they mean}
    \label{bashcommands}
    %make the main header
    \\ \hline
    \textbf{Command} & \textbf{What it does}\\ \hline \hline
    \endfirsthead
    %make the next page header
    \hline
    \multicolumn{2}{||c||}
    {\tablename\ \thetable\ -- \textit{Continued from previous page}}
    \\ \hline
    \textbf{Command} & \textbf{What it does}\\ \hline \hline
    \endhead
    %make the previous page footer
    \multicolumn{2}{||c||}
    {\tablename\ \thetable\ -- \textit{Continued on next
        page}} \\ \hline
    \endfoot
    %make the last footer
    \hline
    \endlastfoot
    %contents of the table
    \verb|ls| & list contents of current directory
    \\ \hline
    \verb|ls -a| & show hidden files too \\ \hline
    \verb|ls -altr| & see the last changes made to
    the files in a directory \\ \hline
    \verb|mkdir directory| & make a new directory
    \\ \hline
    \verb|cd directory| & change directory \\ \hline
    \verb|cd .. |& go back a directory \\ \hline
    \verb|cd ../..| & go back two directories \\ \hline
    \verb|cd ~ or cd |& go to your home directory \\ \hline
    \verb|pwd| & print working directory \\ \hline
    \verb|~/| & means your root \\ \hline
    \verb|.| & means the current directory \\ \hline
    \verb|cp file/to/copy where/newName| & copy
    a file \\ \hline
    \verb|cp file/to/copy .| & copy a file to current
    directory without changing the name \\ \hline
    \verb|cp directory/* . |& copy all the files in a
    directory \\ \hline
    & to the current directory \\ \hline
    \verb|cp -r directory new/directory| & copy a
    directory recursively \\ \hline
    \verb|rm file/to/remove| & remove a file \\ \hline
    \verb|rmdir directory| & remove a directory \\ \hline
    \verb|rm -rf directory| & blow away a directory
    permanently \\ \hline
    \verb|mv file/to/move where/newName| & moves
    or renames a file \\ \hline
    \verb|man command| & show the manual for a
    command \\ \hline
    \verb|cat file/one file/two| \textgreater \verb| new_file|
    & concatonate two or more
    files into a new file \\ \hline
    \verb|history| & shows a history of your
    commands \\ \hline
    \verb|less file/to/see| & shows one page of a
    file \\ \hline
    & space turns the page q quits \\ \hline
    \verb|head file/to/see| & see the first page
    of a file \\ \hline
    \verb|head -n 8 file/to/see| & see the first 8 lines
    of a file \\ \hline
    \verb|tail file/to/see| & see the last page of a file
    \\ \hline
    \verb|tail -n 10 file/to/see| & see the last 10 lines
    of a file \\ \hline
    \verb|grep keyword file/to/search| & search a file
    for a keyword and print all the lines with
    that
    keyword
    to the
    screen \\ \hline
    \verb|history |\textbar\verb| grep keyword| & search your history
    for a keyword \\ \hline
    \verb|grep keyword file/to/search|& count
    the occurences of lines with a \\
    \textbar \verb| wc -l| & keyword\\ \hline
    \verb|command |\textbar \verb| less| & pipe the output of a command
    to less \\ \hline
    \verb|command |\textgreater\textgreater\verb| file|
    & append the output of a
    command to a file \\ \hline
    \verb|command |\textgreater \verb| file| & writes the output of the
    command to a file \\ \hline
    \verb|!command| & executes the most recent command
    that starts with the letters you typed \\ \hline
    \verb|echo something| & print something to the screen \\\hline
    \verb|ls -altr| & see when files in the directory were last
                      altered \\\hline
    \verb|sed -i 's/to replace/new| & find and replace a phrase in a
                                      file \\
    \verb|phrase/' file/to/search| & \\ \hline
    \verb|grep -Rl keyword| & recursively search for a keyword and
                              print the file it was found in \\\hline
    \verb|awk '!a[$0]++' file/to/search| & get rid of duplicate lines
    \\\hline
    \verb|echo "phrase" >> file/to/append| & append a phrase to a file \\\hline
  \end{longtable}
\end{center}

\subsection*{Bash Loops}

Here's a table of basic bash loops and logic, and a basic example in
Table \ref{loopexample} that relates to what we do. 

\begin{center}
  \begin{longtable}{||p{4.5cm}|p{6.5cm}||}
    \caption{Loops in bash}
    \label{loops}
    %make the main header
    \\ \hline
    \textbf{Command} & \textbf{What it does}\\ \hline \hline
    \endfirsthead
    %make the next page header
    \hline
    \multicolumn{2}{||c||}
    {\tablename\ \thetable\ -- \textit{Continued from previous page}}
    \\ \hline
    \textbf{Command} & \textbf{What it does}\\ \hline \hline
    \endhead
    % make the previous page footer
    \multicolumn{2}{||c||}
    {\tablename\ \thetable\ -- \textit{Continued on next
        page}} \\ \hline
    \endfoot
    %make the last footer
    \hline
    \endlastfoot
    %contents of the table
    \verb|for i in {1..100}| & for 100 iterations \\
    \verb|do|& \\
    \verb|command $i| & do this thing (\verb|$i|references the index) \\
    \verb|done| & \\
    \hline
    \verb|for i in `ls -d */`| & for every directory in this
                                   directory \\
    \verb|do| & \\
    \verb|cd $i| & cd into every directory \\
    \verb|...| & \\
    \hline
    \verb|if [ condition ]| & check the condition \\
    \verb|then| & if it's true \\
    \verb|command| & do this \\
    \verb|else| & if it's not \\
    \verb|command| & do this \\
    \verb|fi| & \\
    \hline
    \verb|if [ -e file ] |& check if a file exists \\
  \end{longtable}
\end{center}

\begin{center}
  \begin{longtable}{||p{4.5cm}|p{6.5cm}||}
    \caption{Example of a Bash loop}
    \label{loopexample}
    %first header
    \\ \hline
    \textbf{Command} & \textbf{What it does}\\ \hline \hline
    \endfirsthead
    %next page header
    \hline
    \multicolumn{2}{||c||}
    {\tablename\ \thetable\ -- \textit{Continued from previous page}}
    \\ \hline
    \textbf{Command} & \textbf{What it does}\\ \hline \hline
    \endhead
    %footer
    \multicolumn{2}{||c||}
    {\tablename\ \thetable\ -- \textit{Continued on next
        page}} \\ \hline
    \endfoot
    %last page footer
    \hline
    \endlastfoot
    \verb|for i in {1..100}| & for 100 times \\
    \verb|do| & \\
    \verb|cd E.$i| & enter the directory named
    \verb|E.#|\\
    \verb|if [! -e KPOINTS ]| & if KPOINTS doesn't exist
    \\
    \verb|echo $i| & print the directory number \\
    \verb|getKPoints| & run the getKPoints script \\
    \verb|fi|
    \verb|cd ..| & go back one directory \\
    \verb|done|& \\
  \end{longtable}
\end{center}


\section*{Appendix C: Emacs} \label{sec:emacs}

Emacs is a text editor that has a bunch of cool shortcuts you can
learn to make editing documents super easy. On a Mac you can download
Aquamacs, which uses the same commands as Emacs but you can click with
your mouse. \ref{emacs} has a chart of basic Emacs
commands.

\begin{center}
  \begin{longtable}{||p{4.5cm}|p{6.5cm}||}%longtable so it wraps pages
    \caption{Emacs commands and what they mean}
    \label{emacs}
    %first header
    \\ \hline
    \textbf{Command} & \textbf{What it does}\\ \hline \hline
    \endfirsthead
    %next page header
    \hline
    \multicolumn{2}{||c||}
    {\tablename\ \thetable\ -- \textit{Continued from previous page}}
    \\ \hline
    \textbf{Command} & \textbf{What it does}\\ \hline \hline
    \endhead
    %footer
    \multicolumn{2}{||c||}
    {\tablename\ \thetable\ -- \textit{Continued on next
        page}} \\ \hline
    \endfoot
    %last footer
    \hline
    \endlastfoot
    %contents of table
    \verb|emacs path/to/file| & enter emacs editor for
    existing file or creates new file with that name \\ \hline
    ctrl+x ctrl+c y & save and quit a file \\ \hline
    ctrl+x ctrl+c n & quit without saving \\ \hline
    ctrl+w & cut a line \\ \hline
    ctrl+y & paste a line \\ \hline
    ctrl+k & kills the contents of a line \\ \hline
    ctrl+k ctrl+k & kills a whole line \\ \hline
    ctrl+shift+- & undo \\ \hline
    ctrl+u \verb|3 command| & executes the command 3 times \\ \hline
    ctrl+x ctrl+f & find and open a file (at the bottom of the screen)
  \end{longtable}
\end{center}

\section*{Appendix D: Basics of Github} \label{sec:github}

This section is by no means comprehensive. In fact, there's probably a
lot of things missing. If you'd like to actually get good at Github,
you have a lot more work to do. But here's some basics.

The basic idea of Github is that several people can work on the same
code at the same time. It also has version control: if you don't like
the recent changes to your code, you can get rid of them by going back
to an old version of the code. You can have a copy of the code on your
machine to work on, and you can push your changes to the main copy of
the code. Most people, however, don't have their repositories open to
the public to edit. They will have to clear any suggested changes that
you push. But as they make edits, you can pull their copy down to your
machine. Github will warn you if there are any changes you have made
to the code that will be overwritten by copying their changes. It's a
powerful code sharing tool, but takes a little getting used to.

First you need to go make a Github account. Then you can fork (make a
copy of) whatever repository of code that you want to copy and
edit. If you don't want to make edits to the code yourself, you don't
have to make a fork, you can just copy directly from the original
repository each time.
To give your computer (or a remote computer) access to your repositories
on Github, add your computer's public SSH key to Github. How to do
this:

On your command line execute the following command to copy
your public ssh key:
\begin{verbatim}
pbcopy < ~/.ssh/id_rsa.pub
\end{verbatim}
Note: If this didn't work, it means you don't have a public key and will
need to generate one. See the subsection below this for instructions
on how to do that. 

Now you will be able to paste this key into your account on Github. To
do this, go to Settings$\Rightarrow$SSH and GPG keys$\Rightarrow$New
SSH key. Make sure you name it something intuitive (e.g. My Mac,
Marylou, etc.). 

Now you can copy code from Github to your machine. It's
probably a good idea to make a directory to hold all the stuff you're
going to copy with a name you'll recognize later (e.g. \verb|aBuild/|
for aBuild, etc.). Usually these directories are in
\verb|~/codes/|. Now there are several scenarios involving the code
you're copying from Github. Here's a couple and what to do about them:

1. If you just want to make a copy of the code and don't
expect any changes to be made to it (like ever), use the following command:
\begin{verbatim}
git clone git@github.com:user/repository.git
\end{verbatim}
You can do this over and over again, it will just overwrite the code
you copied to your machine. 

2.
a. If you expect to make your own changes to the code, start by making
a fork of the original code on Github. Then add a master copy of your
fork to your machine using the following command:
\begin{verbatim}
git remote add master git@github.com:user/repository.git
\end{verbatim}
2.
b. If you have your own code that you would like to edit and put on
Github for others to see, you will have to make your own repository on
Github, then use the same command as above.

3. If you want to be able to add someone else's updates to the code to
your machine, the convention is to call their repository
\verb|upstream|. You can add read only access to their repository to
your machine with the following command:
\begin{verbatim}
git remote add upstream git@github.com:user/repository.git
\end{verbatim}

Now that you have a copy of the code, here's how to make changes to
the code on your machine, and on Github from your machine. Here's a
list of some basic Git commands. Remember that this is not a
comprehensive list.

\begin{center}
  \begin{longtable}{||p{5.5cm}|p{5.5cm}||}
    \caption{Basic Github commands}
    \label{git}
    %first header
    \\ \hline
    \textbf{Command} & \textbf{What it does}\\ \hline \hline
    \endfirsthead
    %next page header
    \hline
    \multicolumn{2}{||c||}
    {\tablename\ \thetable\ -- \textit{Continued from previous page}}
    \\ \hline
    \textbf{Command} & \textbf{What it does}\\ \hline \hline
    \endhead
    %footer
    \multicolumn{2}{||c||}
    {\tablename\ \thetable\ -- \textit{Continued on next
        page}} \\ \hline
    \endfoot
    %last page footer
    \hline
    \endlastfoot
    \verb|git checkout file/to/update| & make a copy of upstream's
    version of the file on your machine \\ \hline
    \verb|git pull upstream master| & update your master copy
    with upstream's version \\ \hline
    \verb|git status| & tells you the status of each file on your
    machine with Github's copy \\ \hline
    \verb|git add file/to/add| & add your version of the file to the
    commit \\ \hline
    \verb|git commit| & commit the changes
    you added. Will open an editor for you to leave a comment\\ \hline
    \verb|git commit -m "comment"|& commit the changes
    you added and leave a comment \\ \hline
    \verb|git push| & pushes your latest commit to your fork on Git \\ \hline
    \verb|git push master upstream| & push your
    request to add your changes to upstream \\ \hline
    \verb|git log| & See history of most recent commits \\ \hline
  \end{longtable}
\end{center}

Note: Atlassian has a good Git tutorial and reference guide. 

\subsection*{SSH key generation}

%%%%% WE NEED TO FIGURE OUT HOW TO DO THIS AGAIN. CAN'T FIND THE
%%%%% WEBPAGE WE RECEIVED OUR INSTRUCTIONS FROM
See https://help.github.com/en/articles/generating-a-new-ssh-key-and-adding-it-to-the-ssh-agent\#generating-a-new-ssh-key 


\section*{Appendix E: Troubleshooting VASP} \label{sec:VASPsettings} %%%%%TODO
%I lifted this section straight from my internship report. It might
%need some work to make sense in this document....

If you ever want to know more about the settings you can use in Vasp,
you can go to:
https://cms.mpi.univie.ac.at/wiki/index.php/The\_VASP\_Manual.

\vspace{5mm}
\noindent
Default settings used for all calculations: 
\begin{itemize}
  \item{PREC = a }
  \item{LWAVE = False }
  \item{LREAL = auto }
  \item{ISMEAR = 1 }
  \item{SIGMA = 0.1 }
\end{itemize}

\vspace{5mm}
\noindent
DFT+U settings used:
\begin{itemize}
  \item{ENCUT = 550 }
  \item{LWAVE = False }
  \item{ LDAU = True }
  \item{ LDAUTYPE = 1 }
  \item{ LDAUL = 3 -1 -1 }
  \item{ LDAUU = \# 0 0 (\# changes with ramping scheme explained in
      Section \ref{sec:UOchallenges}) }
  \item{ ICHARG = 1 (after the first iteration of ramping scheme
      explained in Section \ref{sec:UOchallenges}) }
  \item{ LDAUJ = 0.51 0 0 }
  \item{ ISMEAR = -5 }
  \item{ LMAXMIX = 6 }
  \item{ ISIF = 2 }
  \item{ NSW = 0 }
\end{itemize}

\vspace{5mm}
\noindent
Settings used to assign magnetic moments: 
\begin{itemize}
  \item{ ISPIN = 2 }
  \item{ MAGMOM = +/- 2 for U, 0 for O }
\end{itemize}
  
\vspace{5mm}
\noindent
In the case of non-convergence, ionic steps were allowed at a setting
of: 
\begin{itemize}
  \item{ NSW = 3 }
\end{itemize}
\noindent
and this additional setting was added: 
\begin{itemize}
  \item{ IBRION = 2 }
\end{itemize}

\vspace{5mm}
\noindent
In the case of the following error: 
\begin{itemize}
  \item{
      \verb|VERY BAD NEWS! internal error in sub-|\\
      \verb|routine SGRCON|}
\end{itemize}
\noindent
The following setting was changed:
\noindent
\begin{itemize}
  \item{ INCLUDEGAMMA = False}
\end{itemize}
and KPOINTS was regenerated. If it still did not work, the following
setting was changed from the default: 
\begin{itemize}
  \item{ SYMPREC = $10^{-4}$ }
\end{itemize}

\vspace{5mm}
\noindent
In the case of the following errors: 
\begin{itemize}
  \item{
    \verb|POSMAP internal error: symmetry equi-|\\ 
    \verb|valent atom not found, you might try |\\
    \verb|decreasing or  increasing SYMPREC by |\\
    \verb|an order of magnitude.|}
  \item{
    \verb|VERY BAD NEWS! internal error in sub-|
    \verb|routine PRICEL (probably precision |\\
    \verb|problem, try to change SYMPREC in |\\
    \verb|INCAR ?): Sorry, number of cells and |\\
    \verb|number of vectors did not agree.|}
  \item{
    \verb|RHOSYG internal error: stars are not |\\ 
    \verb|distinct, try to increase SYMPREC to |\\
    \verb|e.g. 1E-4|}
\end{itemize}
\noindent
The following setting was changed from the default:
\begin{itemize}
  \item{ SYMPREC = $10^{-4}$ }
\end{itemize}

\vspace{5mm}
\noindent
To run calculations in parallel, the following settings were added: 
\begin{itemize}
  \item{ LPLANE = True }
  \item{ NCORE = (number of cores) }
  \item{ LSCALU = False }
  \item{ NSIM = 4 }
\end{itemize}


\bibliographystyle{ieeetr}
\bibliography{refs}

\end{document}


