\documentclass{article}
\usepackage[utf8]{inputenc}
\usepackage{relsize}
\usepackage{float}
\usepackage{longtable}
\usepackage[dvipsnames]{xcolor}
\usepackage{graphicx} %for pictures
\usepackage{eufrak} %for fancy neighborhood n
\usepackage{indentfirst} %to indent the first paragraph after a title
\usepackage{enumitem} %itemize spacing options
\usepackage{textcomp} %for tildesssssss?????!!!
\usepackage{nameref} %for referencing unnumbered sections by name
\usepackage[section]{placeins} %float barriers
\setlist[itemize]{nosep, leftmargin=.5cm, topsep=0pt, partopsep=0pt}
\setlist[enumerate]{leftmargin=.5cm}

\title{aBuild Sheet}

\begin{document}
\maketitle

aBuild commands start with ``\verb|python builder.py **YML**|" and
then some tag(s). This prefix is only used for the tags denoted in
this table by the ``-''. 
    
\begin{center}
  \begin{longtable}{||p{4cm}|p{7cm}||} %longtable so it wraps %
    % pages
    \caption{Algorithm steps and their descriptions.}
    \label{algorithm}
    \\ \hline
    \textbf{Step} & \textbf{Description}\\ \hline \hline
    \endhead
    % make the previous page footer
    \hline
    \multicolumn{2}{||c||}
    {\tablename\ \thetable\ -- \textit{Continued on next
        page}} \\ \hline
    \endfoot
    %make the last footer
    \hline
    \endlastfoot
    %contents of the table
    -enum & enumerates the crystalline structures up to sizes specified in your
            yaml \textcolor{red}{--run interactively}\\ 
    -setup\_relax & Builds to-relax.cfg and relax.ini; runs calc-grade
                    \textcolor{red}{--run interactively}\\
    qsub* jobscript\_relax.sh & mlp relax: needs to-relax.cfg, pot.mtp, and
                                relax.ini; generates: relaxed.cfg,
                                unrelaxed.cfg, and candidates.cfg \textcolor{red}{--job
                    submission, parallel, 10-30 cores, 6-30 hrs}\\
    -setup\_select\_add & Concatenates all of the candidate.cfg\_\#,
                          selection.log\_\#, relaxed.cfg\_\# and
                          unrelaxed.cfg\_\# into one file
                          each. relaxed.cfg file should get bigger and
                          bigger with each iteration. Also builds a
                          submission script. \textcolor{red}{--run interactively}\\
    qsub* jobscript\_select.sh & mlp select-add: generates:
                              new\_training.cfg; needs: train.cfg,
                              candidate.cfg \textcolor{red}{--job
                              submission, single core, 1-4 hrs}\\
    -add & builds A folders in training set and creates jobscript \textcolor{red}{--run interactively}\\
    qsub* jobscript\_vasp.sh & runs vasp calculations for
                                           the selected configurations
                                            \textcolor{red}{--array
                                            job, 6-30 hrs}\\
    -setup\_train & Pulls data from VASP folders, builds
                    train.cfg and pot.mtp \textcolor{red}{--run interactively} \\
    qsub* jobscript\_train.sh & mlp train: needs train.cfg, pot.mtp;
                                generates: Potential.mtp \textcolor{red}{--job
                    submission, parallel, 10-20 cores, 6-12 hrs}\\
   go back to step -setup\_relax & Repeat until model is fully
                                    trained, i.e. all structures relax.\\
  \end{longtable}
\end{center}
*For Marylou use sbatch instead of qsub.
**YML** is the yaml file without the .yml extension.
    

\subsection{Other helpful MTP/aBuild commands}
These might come in handy at some point
\begin{center}
  \begin{longtable}{||p{4cm}|p{7cm}||} %longtable so it wraps %
    % pages   
    \caption{aBuild commands and description}
    \label{bashcommands}
    \\ \hline
    \textbf{Command} & \textbf{Description}\\ \hline \hline
    \endhead
    % make the previous page footer
    \hline
    \multicolumn{2}{||c||}
    {\tablename\ \thetable\ -- \textit{Continued on next
        page}} \\ \hline
    \endfoot
    %make the last footer
    \hline
    \endlastfoot
    % contents of the table
    -status & prints a status report for Vasp calculations \\
    -report & creates data report file from completed Vasp
                           calculations \\
    -report -file path/to/file & creates data report file from the
                           configurations contained in the file specified \\
    -chull -file path/to/file & creates a convex hull from the data
                           report file specified \\
    mlp mindist file.cfg & prints the global minimum distance of the
                           atoms to eachother in a .cfg file. Also
                           adds the mindist attribute to structures in
                           the .cfg file\\
    mlp calc-grade pot.mtp train.cfg train.cfg temp.cfg & creates
                           state.mvs, a file needed for relaxation \\
    
  \end{longtable}
\end{center}

\pagebreak

\section*{VASP files and what they do} \label{sec:vaspinput} %%%TODO
Pretty self explanatory name. If you want to learn a bit more about
VASP, this is a good section to read.

In order to run any Vasp calculation, you need a POSCAR, POTCAR,
INCAR, PRECALC, and KPOINTS. The rest of these files are output
files. There are a couple other output files I haven't mentioned
here. 
%I'll have to come back and add some stuff in here later.... This might
%be better if it wasn't an appendix and just a section in the regular
%ref guide. 

\subsection*{POSCAR}
This is the ``POSition'' car. Not really sure who came up with the
naming convetion, but whatever \verb|\_(-.-)_/|.
In this file there is title. There is also the lattice parameter,
which is what you multiply all the lattice vectors by to get the
cartesian coordinates for the lattice vectos, which are the three
lines that follow. Then there are the atom counts for each species (in
reverse alphabetical order), and then the coordinate system. This can
be ``D'' for direct, meaning that you multiply the first number of the
basis vector by the first lattice vector, the second by the second,
and the third by the third, then add up the x-coordinates, the
y-coordinates, and the z-coordinates to find the cartesian coordinate
of each atom within the unit cell. The ``C'' stands for cartesian
coordinates, so each number is just an x, y, or z component of the
basis atom's position. Next come the basis vectors, in either direct
or cartesian coordinates. 

If you didn't already know, the lattice vectors are the repeating unit
of a crystalline structure. If you slide the origin to the end of one
lattice vector, you will end up on the same position (as in, the same position
but in a neighboring unit cell). The basis vectors are the positions
of all the atoms in the unit cell. Usually there is one basis vector
with a value of (0,0,0), meaning that the lattice vectors lie in the
middle of one of the atoms. 

\subsection*{INCAR}
This is is ``INput'' car. This file has all the settings for the vasp calculations. In
refGuide.pdf I talk about some of the settings we've
used in past works. They're probably safe to use, but you might want
to consult Brother Nelson about this. 

\subsection*{PRECALC}
This is the input file for k point generation. The only thing you need
to touch in here is mindistance, unless you find yourself in some very
dire circumstances. See refGuide.pdf for an explanation
of such dire circumstances...

\subsection*{POTCAR}
This is the ``POTential'' car. It has the psuedopotentials in it. You
can \verb|grep TITEL POTCAR| to make sure that the correct atom types
are in here.   

\subsection*{KPOINTS}
This is generated by the getKPoints script. You probably don't ever
really have to worry about it unless you forget to generate it. 

\subsection*{OUTCAR} 
This is the ``OUTput'' car. It has some of the output in it. You can
\verb|grep TOTEN OUTCAR|  to see the total energy. If the energy has
converged, you will see ``\verb|free  energy|'' with two spaces.

\subsection*{CONTCAR}
This is an output file. If you let the atoms relax, this is the new POSCAR. Kinda. It just
specifies the new positions of the atoms. Also I'm not sure what the
``CONT'' is. 

\subsection*{CHGCAR}
Not sure, maybe the ``CHarGe'' car? Ask Brother Nelson

\subsection*{OSZICAR}
This has how many iterations Vasp has done in it. I'm not sure what
other purpose it serves, and I'm also not sure what the ``OSZI'' is.

\end{document}
